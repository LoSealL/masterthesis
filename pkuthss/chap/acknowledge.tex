% vim:ts=4:sw=4
% Copyright (c) 2014 Casper Ti. Vector
% Public domain.

\chapter{致谢}
% 中文测试文字。
一转眼在北大已经度过了6个年头,第7个年头也已经快要走完。
在曾经带过的地方,也早以``老人''、``骨灰''自称了。
北大陪伴了我无数个日日夜夜,我也见证了新太阳、理教、文科楼群、南门办公楼、公主楼、岩壁、二体、一体的起落。
那个曾经古朴与静谧的北大随着我的成长消失了,越来越现代与喧闹的校园迎接着新一代的到来。

在做13级本科辅导员的时候,还能记得他们刚入校时的稚嫩身影,感慨着年年代沟深。
本科毕业的那一年,看着一体新岩壁落成,然而至今却鲜去攀爬了。
其实自本科毕业后已经很久没有在校园里散步了,科研工作充斥着整个硕士生涯。
每天总是忙忙碌碌,春夏秋冬,走出微纳大厦,总能看见夜里伫立在月色下的博雅塔,亮灯的那一刻提醒着我新的一年的到来。

而今再次漫步在校园中,又是一个春天,还依稀记得静园的玉兰,未名湖的野鸭,和漫天的杨絮。
28楼前的银杏和老岩壁。
世家的煎饼果子和康博斯的麻婆豆腐。
博实的包子大叔和图书馆的蔺师傅。

它们在这三年中渐渐离我远去,簇拥来的是信息办的学工工作、13级的小朋友、CHES 和 DATE 的论文还有无数的拒稿意见。
这三年是一个全新的生活,我收获到了大学以来最宝贵的礼物。

感谢我的导师贾老师对我的引导,感谢张老师对实验室细致的管理。同样也要感谢刘黎师兄,甘善良师兄对我的指导。

感谢通宵供电的实验室——见过凌晨四点的微纳大厦么。

感谢山鹰社的小伙伴——两个冬天雪宝顶的冬攀。

也感谢对我的研究工作和论文提出过任何建议和帮助的朋友。

最后,我终于要离开校园,自己面对生活了。非常感谢北大,给了我宽阔的视野,自由的时间和思想,以及独立深入思考的能力。
7年时间,我度过了同一个中学,也度过了同一个大学。7年时间,走过了大江南北,经历了起伏坎坷。
登了7座雪山,骑了4000公里长线,跑了2个马拉松,流了4个芯片。非常非常怀念着所有的时光,怀着过去的记忆和未来的希望,合上这论文的最后一篇,往前看,那就是我的生活了。

\rightline{于燕园}
