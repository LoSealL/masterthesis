% vim:ts=4:sw=4
% Copyright (c) 2014 Casper Ti. Vector
% Public domain.

\specialchap{序言}
% 中文测试文字。
信息安全自古以来是各方关注的焦点。小,关乎个人隐私、人身财产安全;大,系家国安危。远,有古典军事加密技术、恩尼格码攻防战;近,有斯诺登棱镜门事件。

棱镜门事件的揭露,让各国政府充分认识到机密数据保护的重要性,同时也提出了一个困难的问题:如何才能保证数据的安全呢?

古代,对于机密的军事报文,采用传统的简单映射,将原文字符映射成密文字符,映射的参数就作为密钥由相互信任的若干方保存。敌方截获后,由于映射的不可逆特性,敌方无法在没有密钥的情况下破解密文。但在工业革命之后,随着人类运算能力的显著提高,曾经难以破译的加密方式可以借由穷举暴力破解,正是如此,不断促进了加密解密的协同发展——一项最新的加密技术往往催生出新颖的解密手段,再反过来促进加密的改进,如此往复。同一时期的加密技术的密钥空间往往远大于该时期计算能力,所以破译方总是寻找数学上的辅助手段来降低运算需求,如古典密码的频次分析法,近代的差分分析法、微分分析法,现代的旁道分析、大数据,甚至社会工程学方法。

其中以物理手段获取旁道信息破译的方法尤为引人关注。旁道信息是与密文相关的中间信息,呈现方式有很多种,比如加密操作时的功耗、电磁辐射、热辐射、运算错误、存储介质状态等等。利用这些信息,可以快速的破译一些加密系统。如差分功耗分析法,通过相似操作的功耗差值来判断猜解密钥的正确性,大幅度削减了穷举量;又如错误分析,通过手段注入使加密系统产生运算错误,并根据出现错误的时机和现象筛选密钥;而存储介质的分析则通过电磁探测等手段直接观察存储器中的逻辑值,从而提取出关键信息。

针对每一种不同的攻击手段,必须分别采取防御措施。如针对差分功耗分析,使用随机掩码隐藏真正的功耗信息,增加攻击者破解的成本;针对错误注入,加入探测机制阻止系统在出错的时候暴露关键信息;而针对存储介质的攻击,则可以采用PUF技术隐式存储关键数据。

在PUF技术出现之前,双方通信中关键的密钥往往直接存储在非易失性存储器(如只读存储器ROM)中。比如门禁系统中门卡的RFID,只能保存在门卡自身的芯片内。尽管通信协议可以很好的加密通信信道内的数据,但是却不能保护芯片内部ROM中的信息。一个恶意攻击者一旦获取了一个门卡,则可以通过技术手段探查ROM中的数据提取出关键信息。而PUF则以加密系统的物理实现自身特点存储信息,相较于ROM等传统非易失性存储器,具有隐秘性好,不可探查等特点,因此受到了相关领域研究者的广泛关注。

Physical Unclonable Function(PUF)最早由MIT的理学博士 Pappu S. Ravikanth 于2001年提出,而其最初被称为 Physical One-Way Function。Ravikanth 在论文中提出了利用可测系统的未知物理状态构造一种Hash函数,函数的映射方式由系统的物理特性决定。这种系统必须具有可观测的量以提供输出,同时以人类现有的知识或计算能力不能仿真或计算系统内部的细节,这样攻击者便不会知道系统将提供什么样的输出。Ravikanth 最后用光学系统实现了他的构想。PUF 一词则由同是MIT的 B. Gassend 等人提出,值得一提的是,Gassend 将PUF的全程写作 Physical Random Function,并称为了不和“伪随机函数”(Pseudo-Random Function)混淆,而写作 Physical Unclonable Function,记作``PUF”。 Gassend 真正提出了基于硅基电路实现的PUF,他用一系列双口交换器级联的方式,通过检测输出端口延迟先后,将工艺随机波动转换成电平逻辑输出,他的这种电路随后被称为 Arbiter-PUF。不仅如此, Gassend 还提出了一整套PUF系统的完善措施,包括输入、输出矫正和PUF的实际应用可能,并给出了基于FPGA的实验结果,可以说给后来的研究者奠定了完善的基础和研究模板。

但自21世纪初提出PUF的随后几年里,对于PUF的研究文献寥寥,而此期间硬件安全的相关研究者热衷于研究同样是世纪之交提出的概念“差分功耗攻击”,直到 2007 年之后,对于 DPA 的研究热度开始随着几个定论的提出开始转冷,而随着半导体制程工艺的不断进步,以及计算机计算能力的跨越式提高,才开始慢慢恢复对 PUF 的研究。尤其是 2004 年 MIT 的 Daihyun Lim 的硕士毕业论文,对 Arbiter-PUF 建模,并用支持向量机(SVM)拟合出 Arbiter-PUF 的模型参数,开启了机器学习对 PUF 的建模攻击领域。此后,接着机器学习崛起的东风, PUF 研究在攻防两方的博弈中迅速崛起,近几年来吸引了越来越多的研究者和相关会议关注这一领域。

到目前为止,PUF技术尚未成熟到商用地步,主要有以下几个技术难点。
\begin{itemize}
\item 其一,需要稳定的生成关键数据。系统的物理特征易受环境变化、使用寿命等因素的影响,基于物理特征生成的数据必须克服物理特征波动带来的影响;
\item 其二,安全性。提取的物理特征应不易于被现有技术模拟,不能预测物理特征以破解出关键数据;
\item 其三,实用性。结合上两点,还必须易于实现,以现有技术能在较低成本下制作出来。
\item 最后,PUF并不适合放在现有的安全系统中,因此有越来越多的文献着手搭建以PUF为核心的安全协议,相信随着PUF研究的不断深入以及成熟的安全协议的提出,PUF将会成为安全领域的一颗新星。
\end{itemize}

本文在前人研究基础上,从基本原理入手,分析PUF的建模于仿真,针对机器学习建模攻击的应用和防范,主要研究成果有以下几点:
\begin{itemize}
\item 第一,对已提出的PUF结构进行建模,通过该模型仿真分析其性能指标,成功通过机器学习拟合模型参数;
\item 第二,改进PUF结构,设计新型的电路结构以达到较高安全性,尤其是对建模攻击的防范;
\item 第三,在FPGA上实现改进电路,通过PCI-E接口与PC通信,收集并测试大量的输出数据验证其性能指标。
\end{itemize}

文章分为六个章节,此为序章。第\ref{chap:preliminary}章主要阐述本文设计的基础知识;第三章说明提出的改进型PUF方案;第四章则阐述在FPGA的实现方法;第五章展示实测数据和分析结果;最后第六章进行总结和提出展望。


