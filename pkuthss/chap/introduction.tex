% vim:ts=4:sw=4
% Copyright (c) 2014 Casper Ti. Vector
% Public domain.

\chapter{绪论}
\section{背景介绍}\label{sec:background}
信息安全是人类关注的永恒话题。而现代信息安全的大厦是由各种密码协议砌成的,这座大厦的基础就是构筑协议的数学原理。

古典密码学采用的简单映射,如凯撒密码,仿射密码,将原文字符通过线性变换映射成密文字符。由于线性变换的对称性,使得求解逆函数非常容易,这让简单映射密码非常容易受到攻击。最早的非线性加密手段——一次一密密码本——通过不断更换密码表(一种简单哈希表)达到非线性化的目的,只要密码本换的足够勤,那理论上就是不可破解的算法。

现代密码学运用了更深刻的数学知识。以对称加密算法 AES 为例, 128bit AES 加密需要10轮,每一轮都要经过一个非线性运算——S盒——求多项式在$GF(2^n)$上的逆。
求逆之后,密钥的每一位都会与明文发生作用,这便使得逆向求解(已知密文、明文求解密钥)非常困难\supercite{rijmen2001advanced}。
又来看经典公钥算法 RSA,其用两个大质数分别作为公钥和私钥,利用一种单向函数``大整数的因式分解''使正向运算(求大质数的乘积)简单,而逆向运算(求大整数的因数)非常困难来保证安全性。
由此可见,单向函数( One-Way Function )是密码学的基础之一。
1945年克劳德·香农( Claude E Shannon )在其经典著作《密码学的数学原理》\supercite{shannon1945mathematical} 指出密码设计的两个原则:扩散( Diffusion )和扰乱( Confusion ),这两个原则也是构造单向函数的原则。
扩散指输入变化 1bit 会使输出变化 50\% 的 bits,扰乱指输出的每一个 bit 都是由尽可能多的输出 bit 决定。

在众多的密码系统中,单向函数往往由纯数学概念构造,比如大整数因式分解( RSA )、椭圆曲线求解( ECC )、矩阵分解\supercite{wendt2013bidirectional}( DBF )等等,在实际的电路实现上往往硬件规模较大,或是求值时间较长。
而现今有越来越多的便携设备接入网络,在公共场合下有通信保密的需求。但是传统的加密方式不适合应用在这些设备上,需要更加低功耗、低硬件开销的解决方案。

便携设备不同于传统主机,流动性大,使得其本身容易被攻击者获取(比如丢失、遭克隆等等手段,而传统攻击者获取的仅仅是信道信息),这使得以物理手段获取旁道信息破译的方法在近几年尤为引人关注\supercite{standaert2010introduction}。
旁道信息是与密文相关的中间信息,有多种呈现方式,比如加密操作时的功耗、电磁辐射、热辐射、运算错误、存储介质状态等等。利用这些信息,可以快速的破译一些加密系统。
如差分功耗分析法,通过相似操作的功耗差值来判断猜解密钥的正确性,大幅度削减了穷举量;
又如错误分析,通过手段注入使加密系统产生运算错误,并根据出现错误的时机和现象筛选密钥;
而存储介质的分析则通过 多种手段直接观察存储器中的逻辑值,从而提取出关键信息。比如针对非易失性存储器(Non-volatile Memory, NVM),通过侵入式或非侵入式方法可以读出其中的存储信息,如果关键密钥也是放在当中,则可能会被读取出从而获得系统权限\supercite{helfmeier2013cloning}。
比如门禁系统中门卡的 RFID,只能保存在门卡自身的芯片内。尽管通信协议可以很好的加密通信信道内的数据,但是却不能保护芯片内部 ROM 中的信息。一个恶意攻击者一旦获取了一个门卡,则可以通过技术手段探查 ROM 中的数据提取出关键信息。

针对每一种不同的攻击手段,必须分别采取防御措施。如针对差分功耗分析,使用随机掩码隐藏真正的功耗信息,增加攻击者破解的成本\supercite{rivain2010provably};针对错误注入,加入探测机制阻止系统在出错的时候暴露关键信息\supercite{karaklajic2013hardware};而针对存储介质的攻击,则可以采用 PUF 技术隐式存储关键数据。

在 PUF 技术出现之前,采用物理手段或者特殊工艺防止对存储器的非法读取,尤其是限制反向工程。但这样做会大大增加芯片成本,往往只用于一些高度保密的关键芯片中。
而 PUF 则以加密系统的物理实现自身特点存储信息,相较于传统保护手段,功耗小、成本低,因此受到了相关领域研究者的广泛关注。
而且随着研究深入,人们发现 PUF 的特点也可以使其应用于其他加密场景,比如 IP 保护\supercite{Guajardo2007fpga,kumar2008butterfly},不经意传输协议\supercite{ruhrmair2010oblivious},密钥交换\supercite{rostami2014robust},比特承诺\supercite{ruhrmair2012practical}等等。

\section{PUF相关研究}\label{sec:PUF-research}
Physical Unclonable Function( PUF )最早由 MIT 的理学博士 Pappu S. Ravikanth 于 2001 年提出\supercite{pappu2002physical},而其最初被称为 Physical One-Way Function 。
Ravikanth 在论文中提出了利用可测系统的未知物理状态构造一种 Hash 函数,函数的映射方式由系统的物理特性决定。
这种系统必须具有可观测的量以提供输出,同时以人类现有的知识或计算能力不能仿真或计算系统内部的细节,这样攻击者便不会知道系统将提供什么样的输出。
Ravikanth 最后用光学系统实现了他的构想。
PUF 一词则由同是 MIT 的 B. Gassend 等人提出\supercite{gassend2002silicon}。
值得一提的是, Gassend 将 PUF 的全称写作 Physical Random Function,并且称为了不和``伪随机函数''( Pseudo-Random Function )混淆,而写作 Physical Unclonable Function,记作``PUF''。
Gassend 提出了基于硅基电路实现的 PUF,他用一系列双口交换器级联的方式,通过检测输出端口延迟先后,将工艺随机波动转换成电平逻辑输出,他的这种电路随后被称为 Arbiter-PUF。
不仅如此, Gassend 还提出了一整套 PUF 系统的完善措施,包括输入、输出矫正和 PUF 的实际应用可能,并给出了基于 FPGA 的实验结果,可以说给后来的研究者奠定了完善的基础和研究模板。

自硅基(集成电路) PUF 提出后, PUF 的研究逐渐演化为几个方向。
其一是 PUF 安全性的研究,通过理论和实验提取 PUF 模型参数;
其二是 PUF 电路结构的研究,追求更高的可靠性,低功耗和高集成度;
其三是 PUF 的应用领域,构建基于 PUF 的密码协议。

2004 年 MIT 的 Daihyun Lim 在其硕士毕业论文\supercite{lim2005extracting}提出对 Arbiter-PUF 建模,并用支持向量机( SVM )拟合出 Arbiter-PUF 的模型参数,开启了机器学习对 PUF 的建模攻击领域。
2007 年 Guajardo 在 \parencite{Guajardo2007fpga} 中提出 ``Weak'' 和 ``Strong'' 的概念,此后人们接受并将 PUF 按此标准人为分成两类: Strong PUF 和 Weak PUF,它们相关定义和具体区别见\ref{subsec:weakpuf}小节。 Weak PUF 多是基于 SRAM 类型,应用于密钥生成,芯片指纹等场合; Strong PUF 主要有基于延迟和基于环振两种类型,应用场合比 Weak PUF 丰富许多,但是 Strong PUF 易受到建模攻击,其根本原因是由于其输入-输出对(CRP)的秩不够大,容易根据一部分子集推算出剩余未知 CRP ,这就是建模攻击的基本原理。
针对这一点, 研究者开始评价 Strong PUF 的非线性特点。 Katzenbeisser 等人在\parencite{katzenbeisser2012pufs}中评价了多种现有 PUF 的安全特性,其中属于 Strong PUF 的 Arbiter PUF 和 RO PUF 的信息熵\footnote{信息熵反应信息流的内容含量,单位是比特。文中用的是条件信息熵}都比较低。
在提升 Strong PUF 熵的研究上,有 Suh 的异或 PUF\supercite{suh2007physical}, Majzoobi 等人的 Lightweight PUF \supercite{majzoobi2008lightweight}, Chen 的 Bistable Ring PUF\supercite{chen2011bistable},还有 Kumar 和 Burleson 的电流镜 PUF\supercite{kumar2014design}。

然而,建模攻击所利用的机器学习算法,以及并行计算,分布式计算等技术在近几年同样得到了长足发展,使得研究者可以在相对很短的时间内进行推算,并且处理大量数据。
因此可以将 Strong PUF 进行如下分类:已建模、尚未建模和不可建模。已建模的 PUF 可以按建模攻击的运算时间评价好坏;尚未建模的 PUF 需要寻找到合适的模型或者证明其不可建模;不可建模的 PUF 理论上则是完全不受建模攻击威胁,但目前为止没有人提出过不可建模 PUF。
Mahmoud 等人用分布式集群对 Lightweight, Arbiter 和 XOR 三种 PUF 做了建模攻击和时间统计\supercite{mahmoud2013combined,xu2014hybrid},已建模的 PUF 原则上一定会被机器学习拟合出模型参数,但是越复杂的模型拟合时间越长,如果超出了计算能力,则可以认为这样的 PUF 也可以抵御建模攻击。
但是 Mahmoud 利用旁道攻击方法对复杂模型进行中间值提取,从而降低了模型复杂度,大大缩短了计算时间。
所以传统的异或、前馈等方式不能从根本上解决建模攻击的威胁。

到目前为止, PUF 技术尚未成熟到商用地步,主要有以下几个技术难点。
\begin{itemize}
\item 其一,需要稳定的生成关键数据。系统的物理特征易受环境变化、使用寿命等因素的影响,基于物理特征生成的数据必须克服物理特征波动带来的影响;
\item 其二,安全性。提取的物理特征应不易于被现有技术模拟,不能预测物理特征以破解出关键数据;
\item 其三,实用性。结合上两点,还必须易于实现,以现有技术能在较低成本下制作出来。
\item 最后, PUF 并不适合放在现有的安全系统中,因此有越来越多的文献着手搭建以 PUF 为核心的安全协议,相信随着 PUF 研究的不断深入以及成熟的安全协议的提出, PUF 将会成为安全领域的一颗新星。
\end{itemize}

\section{本文主要工作和文章结构}\label{sec:mainworks}
双稳环路 PUF 是 Chen 在\parencite{chen2011bistable}提出的新型 Strong PUF,尚未被建模,但已有的分析工作指出 BRPUF 的扩散(Diffusion)性不好,以及输出分布不均\supercite{chen2012characterization,yamamoto2014security}。
Schuster 等人在\parencite{schuster2014evaluation}中使用单层神经网络对BRPUF进行建模攻击并得到了90\%的预测率。
本文在前人研究基础上,从基本原理入手,分析 BRPUF 的电气原理,提出了精确描述模型,成功对其进行了建模攻击并达到了99\%以上的预测率。
针对机器学习建模攻击,提出了新型结构——基于随机脉冲的PUF。主要研究成果有以下几点:
\begin{itemize}
\item 第一,理论分析。对已提出的 BRPUF 结构进行建模,通过该模型仿真分析其性能指标,成功通过机器学习拟合模型参数;
\item 第二,电路结构。改进 PUF 结构,设计新型的电路结构以达到较高安全性,尤其是对建模攻击的防范;
\item 第三,仿真与实验。在 FPGA 上实现改进电路,通过 PCI-E 接口与 PC 通信,收集并测试大量的输出数据验证其性能指标。
\end{itemize}

文章分为6个章节,此为序章。第\ref{chap:preliminary}章主要阐述本文设计的基础知识;第\ref{chap:buildingmodel}章说明建模方法和安全性理论证明;第\ref{chap:dbrpuf}、\ref{chap:rpapuf}章则阐述本文提出的两种改进结构以及在 FPGA 的实现方法和实验结果;最后第\ref{chap:conclusion}章进行总结和提出展望。
