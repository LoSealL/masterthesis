% vim:ts=4:sw=4
% Copyright (c) 2014 Casper Ti. Vector
% Public domain.

\begin{cabstract}
信息安全在现代系统设计中占据重要位置,而单向函数是实现安全模块的数学基础之一。目前基于软件的单向函数实现求值时间长、系统开销大,不利于集成在便携式芯片中。
物理不可克隆函数(Physical Unclonable Function, PUF)是以硬件方式实现单向函数,利用硬件实现的单向函数求值迅速、结构简单、集成度高、成本低廉,使得 PUF 适合于有安全性需求的移动端芯片。

本文研究了 PUF 在电路系统的实现方法,论文的主要工作和创新点包括:

第一,
提出了一种精确描述双稳态环路型物理不可克隆函数 (BRPUF) 的数学模型,该模型能够准确表达 BRPUF 的输出状态。利用该模型对 BRPUF 的建模攻击可以用 5000 个数据点达到 99\% 的预测率。

第二,
提出了延迟型双稳态型物理不可克隆函数 (DBRPUF) 设计方案,其结合了仲裁型 PUF 和 BRPUF 的优点,改进了 BRPUF 响应存在偏置的不足,且在可编程门阵列 (FPGA) 上实现了电路并采样验证。

第三,
设计了随机脉冲采样 PUF (RPAPUF),该结构利用短脉冲传播的不稳定特性,引入随机变量,使攻击者难以建模预测或仿真电路行为,增强了其抵御建模攻击的能力;
同时通过真负边沿采样异或运算,增加了建模的空间复杂度,使得攻击者需要极大量的运算代价对此结构进行建模。

最后本文比较了新提出结构在内的多个 PUF 的统计分布、 NIST 测试结果和建模攻击预测结果。实验结果显示本文提出的 DBRPUF 的随机性标准差为0.1,是 BRPUF 的33.4\%,独特性标准差为0.03,是 BRPUF的15.1\%;而 RPAPUF 的随机性标准差小于0.1,独特性标准差小于0.02,对其建模攻击的预测率不超过67\%。
	
\end{cabstract}

\begin{eabstract}
Modern system design concerns more and more about security. As one of the basic foundation of crypto-protocols, one-way function and its implementation takes efforts to do so. Physical Unclonable Functions (PUF) is an alternative to one-way function. PUF utilizes the physical system which is unknown to mankind to set up a projection from challenges to responses. This physical-disorder based system, which has low energy consumption and high density, is quite suitable for portable security chips.

In this thesis, I analyzed and implemented PUF in Field Programmable Gate Array (FPGA), and investigated the randomness, uniqueness and reliability of the Arbiter, Bistable-Ring and XOR PUFs.
I also built an accuracy models for Bistable Ring PUF and extract characteristics from the model. The modeling attack to BRPUF via this model archives 99\% prediction rate.
This thesis proposed two novel PUF designs. One combines the advantage of the Arbiter PUF and the Bistable Ring PUF. The experiment on FPGA demonstrated that it can improve the randomness and uniqueness compared to the BRPUF.
The other one, referred to as Random Pulse Arbiter PUF (RPAPUF), introduces a random bit with a random input pulse signal to confuse the attacker who wants to predict it. The experimental results indicate the proposed RPAPUF has a significant increment on the space and the time complexity of building models for it.

In the end, this thesis summarized the statistical and NIST test results and modeling attack results for all mentioned PUFs.

\end{eabstract}

