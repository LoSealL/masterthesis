% vim:ts=4:sw=4
% Copyright (c) 2014 Casper Ti. Vector
% Public domain.

\chapter{结论}\label{chap:conclusion}
\section{全文总结}
在密码学研究中,单向函数是构造高层次密码协议的基础。 PUF 作为物理不可克隆函数,是基于人类不可知的物理原理构造的单向函数,在密码学领域有着重要应用。

本文基于 Strong PUF 的建模,根据研究者的已有工作,对 BRPUF 进行了精确建模,通过 Matlab 仿真和 FPGA 验证,证明了模型的准确度,其建模攻击预测率超过99\%只需要不到5000个训练数据点。
其次,本文从数学原理出发,用统计方法分析了 Arbiter PUF, BRPUF 和 XOR-PUF 的输出概率分布,解释了 BRPUF 的不稳定性和偏移原因:
因为 BRPUF 在求值过程中累积了所有逻辑门的延迟偏差,导致结果的统计方差大大增加。

为了消除 BRPUF 的系统偏差,提升 PUF 的安全性,本文提出了两种新型 PUF 结构。
其中第\ref{chap:dbrpuf}章的 DBRPUF 针对 BRPUF 进行改进,消除了延迟偏差累积,因此改善了统计测试结果,但相比传统异或结构仍显不足;
第\ref{chap:rpapuf}章新提出的随机脉冲信号的 RPAPUF,利用逻辑门的双边延迟构造出复杂的结构,随机选择脉冲输入,利用脉冲传递路程长短``欺骗''攻击者,使其不能分辨出哪些响应符合模型计算结果,哪些是随机项,从而使建立模型需要花费极大开销,相对传统结构大大加强了 PUF 对于建模攻击的抵御能力。

在实验上,为了缩短仿真时间,本文利用 HSPICE 和 Matlab 结合的方法,HSPICE 负责仿真基本逻辑门的工艺波动,Matlab 负责建立由多个逻辑门组成的电路结构,并通过简化模型静态仿真电路行为,使得多比特 PUF 可以在有限时间内仿真完成。
因此本文验证了所有 PUF 结构,并在 Altera FPGA 上实现并比较了 Arbiter PUF, BRPUF, XOR-PUF 和 DBRPUF 与 RPAPUF 结构。
实验结果证实了本文的理论分析,各 PUF 实验分布和理论分布一致。并且 SVM 建模攻击结果证明了本文提出的 RPAPUF 结构不能使用 XOR 模型建模,且难以在有限时间内用精确模型进行建模攻击。

最后,本文研究仍存在一些不足之处。
首先,本文采用的统计分析基于工艺波动服从正态分布这一假设,但是在不同工艺,尤其是深亚微米制程下,这一假设不一定满足,因此不同工艺对于 PUF 的输出影响需要进一步的实验验证。
其次,FPGA 的实现存在布线偏差,且实验用 FPGA 数量不足,不能很好的反应不同 PUF 之间的片间分布差异,因此更精确的验证应该采用全定制方法设计 PUF。
最后,本文尚没有对 PUF 的稳定性进行验证,包括电压波动稳定性、环境温度波动稳定性、和时间稳定性。

我们希望在后续的研究中将提出的 RPAPUF 应用于高层次密码协议,从而分析 PUF 的实用性开销和代价,并通过流片进一步验证新型 PUF 结构在不同工艺下的表现,以及在不同电源电压和环境温度下稳定性表现。


\section{前景展望}

PUF 的安全性始终是 PUF 的终极评价标准,而 PUF 的安全性设计和针对这项设计的攻击一直以来都是彼此的博弈。
目前,侵入式探查、旁道攻击等崭新方式给 PUF 带来更多的设计挑战,而全新的工艺技术和期间技术又给 PUF 设计带来了新的可能\supercite{delvaux2013side,merli2013localized,helfmeier2013cloning}。
不仅有双值逻辑的数字电路 PUF,还有基于 RRAM、模拟电路实现的 PUF 具有更高的非线性特点\supercite{liu2015experimental,chen2015utilizing}。
这些都使得未来的研究非常有趣,无论攻防哪一方占得先机,都会给这一领域的研究带来新的启迪和方向。

再者,安全应用的最终实现还是种种密码协议。
基于 PUF 的密码协议越来越多的涌现出来,比如有基础协议``不经意传输协议''的 PUF 实现\supercite{ruhrmair2010oblivious}, FPGA 上的 IP 保护\supercite{kumar2008butterfly}, ``比特承诺协议''\supercite{ruhrmair2013practical}等等。
尽管 PUF 有着天然的安全属性,但是协议的复杂性使基于 PUF 的协议构建非常困难,尤其是每一篇新的协议模型总会伴随而来一篇攻击方法\supercite{ruhrmair2013pufs}。
在未来,相信随着 PUF 研究的深入,能够找到一种完美利用 PUF 特点,真正实现高效、低功耗、不可克隆的密码协议,使得 PUF 应用于千千万万移动端安全芯片,真正为千家万户的安全护航。