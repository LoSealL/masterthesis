% vim:ts=4:sw=4
% Copyright (c) 2014 Casper Ti. Vector
% Public domain.

\specialchap{结论}\label{chap:conclusion}

在密码学研究中,单向函数是构造高层次密码协议的基础。 PUF 作为物理不可克隆函数,是基于人类不可知的物理原理构造的单向函数,在密码学领域有着重要应用。

本文从数学原理出发,用统计方法和机器学习建模攻击对仲裁型 PUF、异或 PUF、双稳态环路 PUF 进行分析。本文通过模型公式对 Strong PUF 的安全性原理进行了总结,分析并指出了异或结构的安全性原理,双稳态环路 PUF 的统计性失准的原因。根据建模分析,在模型线性表达式或近似线性表达式中,维度的大小决定了机器学习的拟合时间代价。

为了提升 PUF 的安全性,本文提出了两种新型 PUF 结构,其中第\ref{chap:dbrpuf}章的 DBRPUF 针对 BRPUF 进行改进,大大改善了统计结果,但相比传统结构仍显不足;第\ref{chap:rpapuf}章新提出的随机脉冲信号的 RPAPUF,利用逻辑门的双边延迟构造出复杂的结构,随机选择脉冲输入,利用脉冲传递路程长短``欺骗''攻击者,使其不能分辨出哪些 CRP 符合模型计算,哪些是随机项,从而时建立模型需要极大开销,相对传统结构大大加强了 PUF 对于建模攻击的抵御能力。

本文利用 HSPICE 和 Matlab 结合的方法仿真验证了所有 PUF 结构,并在 Altera FPGA 上实现并比较了 Arbiter PUF,BRPUF 和 DBRPUF 结构。结果证实了理论分析,证明了本文提出结构的可行性。

事实上,本文采用的统计分析基于工艺波动服从正态分布这一假设,在不同工艺,尤其是深亚微米制程下,这一假设不一定满足,不同工艺对于 PUF 的输出影响需要进一步的实验验证。

最后,基于 PUF 的密码协议越来越多的涌现出来,我们希望在后续的研究中将提出的 RPAPUF 应用于高层次密码协议,从而分析 PUF 的实用性开销和代价,并通过流片进一步验证新型 PUF 结构在不同工艺下的表现,以及在不同电源电压和环境温度下稳定性表现。相信随着 PUF 研究的深入,能够为传统密码学开启一个崭新的空间。



